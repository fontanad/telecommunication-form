\documentclass[a4paper]{article}
\usepackage[T1]{fontenc}
\usepackage[utf8]{inputenc}
\usepackage[italian]{babel}
\usepackage{multicol}
\usepackage{color}
\usepackage{mathtools}
\usepackage{mathrsfs,amsmath}
\usepackage{geometry}
\usepackage{steinmetz}
%margins
\geometry{a4paper,top=0cm,bottom=0.1cm,left=0cm,right=0cm,
    heightrounded,bindingoffset=0mm} 
%column lines
\setlength{\columnseprule}{1pt}
\def\columnseprulecolor{\color{black}}
%begin of the real document.
\begin{document}
\title{FORMULARIO TELECOMUNICAZIONI}
\author{(Ordinati per cognome) Fontana Davide, Galiazzo Riccardo, Giachelle Fabio, Querinuzzi Giorgio }
\maketitle
\textbf{Attenzione}
Questo formulario è rilasciato con licenza GPL per cui questo formulario non 
può essere venduto. Al contrario può essere liberamente distribuito e ampliato 
rilasciando il sorgente sempre con licenza GPL\@. \\
Si invitano tutti quelli che ne sono interessati ad aggiungere formule 
e/o correggere eventuali errori nel formulario.
Vi ringraziamo in anticipo per i successivi contributi e ci auguriamo
che questo lavoro possa servire a tutte quelle persone che come noi hanno 
perso una montagna di tempo a reperire a tutte le formule.\\
\textit{Gli Autori}
\newpage
%beginning of the multicolon document.
%Change the number for change the number of the colons.
\begin{multicols*}{2} 
%Preliminari
\textbf{Preliminari} \\
Eulero; \\
$\cos(y)=\frac{e^{jy} + e^{-jy}}{2}$; \\
$\sin(y)=\frac{e^{jy} - e^{-jy}}{2}$; \\
$e^{j\alpha}=\cos(\alpha)+j\sin(\alpha)$ \\
$|x(t)|^2=x(t)\cdot x^*(t)$\\
$|x(t)|=\sqrt{{Re[x]}^{2} + {Im[x]}^{2}}$ \\
$\phase{Z_1 Z_2} = \phase{Z_1}+\phase{Z_2}$ \\
$\phase{Z}=\arctan(\frac{Im[Z]}{Re[Z]})$ \\
\underline{\textit{Tempo Continuo}} \\
Area $\int_{-\infty}^{+\infty} x(t)$; \\
Energia $\int_{-\infty}^{+\infty} |x(t)|^2$; \\
Potenza 
$\lim_{T \to +\infty}\frac{1}{2T}\int_{-T}^{T}|x(t)|^2$; \\
\underline{\textit{Potenze Note}} \\
$P_x=\frac{{A}^{2}}{2}$ di segnale sinosuidale\\
\underline{\textit{Tempo Discreto}} \\
Area $\sum_{n=-\infty}^{+\infty} x(nT) \cdot T$; \\
Energia $\sum_{n=-\infty}^{+\infty}|x(nT)|^2\cdot T$; \\
Potenza
$\lim_{N \to +\infty}{\frac{1}{2N + 1}\int_{n=-N}^{+N}|x(nT)|^2\cdot T}$; \\
\underline{\textit{Serie}} \\
\textit{Serie Geometrica}\\
$\sum_{k=N_1}^{N_2}a^k=\frac{a^{N_1}-a^{N_2 + 1}}{1-a}$\\
%Segnali
\textbf{Segnali}; \\
$rect(\frac{t}{T_A})\cdot rect(\frac{1}{T_B})=rect(\frac{1}{\min(T_A,T_B)})$ \\
$sinc^2(0) = 1$;\\
%Tasformate di fourier
\textbf{Fourier} \\
\underline{\textit{Tempo Continuo}} \\
$X(f) = \int_{-\infty}^{+\infty} x(t) e^{-j2\pi ft} dt$; \\
\underline{\textit{Proprietà}} \\
$ax(t) + by(t) \xrightarrow{\mathscr{F}} aX(f) + bY(f)$; \\
$x(-t) \xrightarrow{\mathscr{F}} X(-f)$; \\
$x^{*}(t) \xrightarrow{\mathscr{F}} X^{*}(f)$; \\
$x(t{\pm}t_0) \xrightarrow{\mathscr{F}} X(f)e^{{\pm}j2{\pi}f{t_0}}$; \\
$x(kt) \xrightarrow{\mathscr{F}} \frac{1}{|k|}X(\frac{f}{k})$; \\
$\frac{1}{|k|}X(\frac{t}{k}) \xrightarrow{\mathscr{F}} X(kf)$; \\
$x(t)e^{{\pm}j2{\pi}{f_0}t} \xrightarrow{\mathscr{F}} X(f{\mp}f_0)$; \\
$x(t)\cos(2{\pi}{f_0}t) \xrightarrow{\mathscr{F}} \frac{1}{2}[X(f - f_0) + X(f + f_0)]$; \\
$\int_{-\infty}^{+\infty}x(\tau)y(t - \tau) d\tau \xrightarrow{\mathscr{F}} X(f)Y(f)$; \\
$x(t)y(t) \xrightarrow{\mathscr{F}} \int_{-\infty}^{+\infty}X(a)Y(f - a) da$; \\
\underline{\textit{Trasformate Continue Note}} \\
$rect(t) \xrightarrow{\mathscr{F}} sinc(f)$; \\
$A \xrightarrow{\mathscr{F}} A\delta (f)$; \\
$e^{j2{\pi}f_{0}t} \xrightarrow{\mathscr{F}} \delta (f-f_0)$; \\
$A\cos(2{\pi}f_{0}t) \xrightarrow{\mathscr{F}} {\frac{A}{2}}[\delta(f-f_0) + \delta(f+f_0)]$; \\
$A\sin(2{\pi}f_{0}t) \xrightarrow{\mathscr{F}} {\frac{A}{2j}}[\delta(f-f_0) - \delta(f+f_0)]$; \\
$sinc(t) \xrightarrow{\mathscr{F}} rect(f)$; \\
$sinc^2(t) \xrightarrow{\mathscr{F}} triangle(f)$; \\
$K\delta(t) \xrightarrow{\mathscr{F}} K$; \\
$e^{-\pi{t^2}} \xrightarrow{\mathscr{F}} e^{-\pi{f^2}}$; \\
\underline{\textit{Tempo Discreto}} \\
$X(f) = \sum_{k=-\infty}^{+\infty}{Tx(nT)e^{-j2\pi fnT}}$; \\
\underline{\textit{Trasformate Discrete Note}} \\
$\delta(nT) \xrightarrow{\mathscr{F}} 1$; \\
$\delta(nT - kT) \xrightarrow{\mathscr{F}} e^{-j2{\pi}{f_0}kT}$; \\
$x_n = d_n \xrightarrow{\mathscr{F}} \sum_{n=-\infty}^{+\infty}T\cdot x_n\cdot e^{-j2\pi fnT}$ \\
dove $d_n$ valore discreto in un punto nT \\
$x(nT{\pm}n_0T) \xrightarrow{\mathscr{F}} X(f)e^{{\pm}j2{\pi}f{n_0T}}$; \\
%Convoluzione
\textbf{Convoluzione} \\
\underline{\textit{Tempo Discreto}} \\
$\sum_{k=-\infty}^{+\infty}{Th(nT-Kt)x(nT)}$; \\
\underline{\textit{Tempo Continuo}} \\
$z (t)=\int_{-\infty}^{+\infty}{x(\tau)y(t-\tau)}$\\
\underline{\textit{Convoluzioni Note}} \\
\textit{Triangolo }\\
$k\cdot triangle(t\cdot \alpha) = rect(\alpha t) * rect(\alpha t)$; \\
Altezza Triangolo = k; \\
k=$\frac{1}{\alpha}$
Area (triangle) = Area[rect] x Area[rect]; \\ 
$triangle(t) = rect(t) * rect(t)$; \\
\textit{caso rect diversa ampiezza (Trapezio)} \\
base maggiore=somma base dei due rect\\
baseMinore=baseRectMax---baseRectMin\\
Area[Trapezio]= ((baseMax-baseMin)$\cdot$h)/2\\
\textit{Convoluzione con delta} \\
$X(f)*\delta(f \pm f_0)=X(f\pm f_0)$ \\ 
\textit{Teorema di Parseval }\\
$\int_{-\infty}^{+\infty}{x(t)\cdot y^*(t)dt = }$ 
$\int_{-\infty}^{+\infty}{X(f)\cdot Y^*(f)df}$ \\
%Modulazione 
\textbf{Modulazione} \\
$F_c>=2B$frequenza minima campionamento (B = 1/T)\\
%Campionamento e interpolazione
\textbf{Campionamento e interpolazione} \\
Aliasing presente se $X(f)$ è a banda $\infty$. \\
Se $X(f)$ è a banda finita si può evitare aliasing.\\
Per evitare aliasing si prende $F_c >= 2\cdot B_x$ \\
\underline{\textit{Campionamento}}\\
$Y(f)=\sum_{k=-\infty}^{+\infty}{X(f-\frac{K}{T})}$ \\
\underline{\textit{Interpolazione}}\\
$y(t)=\sum_{n=-\infty}^{+\infty}{T\cdot x(nT)\cdot g(t-nT)}$ \\
%Variabili aleatorie
\textbf{Variabili Aleatorie}; \\
Varianza $\sigma^2=M_x-m_x^2$ \\
$M_x=E[x^2]=\int_{-\infty}^{+\infty} a^2 f_x(a) da$ (potenza statistica) 
e E = aspettazione\\
$m_x=\int_{-\infty}^{+\infty} a f_x(a) da$; (media)\\
%Conversione Analogico Digitake
\textbf{Conversione ADC} \\
$L=2^b$ livelli quantizzati; \\
${\Lambda_q}$ SNR di quantizzazione; \\
$\Lambda_q = L^2$ \\
${({\Lambda}_q)}_{dB}=10\log_{10}{(\Lambda_q)}$
${(\Lambda_q)}_{dB} = 6b$ b numero di bit; \\
${\Lambda}_q=\frac{M_s}{M_{e_q}}$ ($M_{e_q}$ potenza statistica errore
,$M_s$ potenza statistica segnale); \\
$P_{sat}=2Q(\frac{V_{\max}}{\sigma_x})$ (probabilità saturazione) \\
$V_{\max}$ dinamica o valore con meno probabilità d'errore \\
$R_b=F_c \cdot bit$ con $R_b$ bit rate; \\
\underline{\textit{Dinamica Normale}} \\
$V_{\max} = 4\sigma_a$ con $\sigma_a$ deviazione standard;\\ 
\underline{\textit{$M_s$ note}} \\
sinc: $M_s = \frac{V_0^2}{2}$ \\
%Processi Aleatori
\textbf{Processi Aleatori}; \\
$r_x(t)$ correlazione (discreta o continua) \\
\textit{Densità spettrale di potenza} \\
$r_x(f) \xrightarrow{\mathscr{F}} R_x(f)$; \\ 
$R_x(f)$ densità spettrale di potenza; \\
$R_y(f) = R_x(f) |H(f)|^2$ per un processo aleatorio filtrato\\
N.B\@:Il rumore viene applicato all' ingresso del filtro in recezione. \\
\textit{Densità spettrale di potenza media} \\
$\overline{R_y}(f) = R_x(f) |\frac{G(f)}{T}|^2$, $G(f)$ impulso fondamentale\\
$\overline{m_y}(a)=m_a\cdot H(0)$ (media del processo)\\
\textit{Potenza statistica del processo aleatorio} \\
$M_y = r_y(0)$ (continua o discreta) oppure \\
$M_y = \int_{-\infty}^{+\infty} R_y(f)$;\\
$M_y$ = Area del segnale in frequenza del processo aleatorio \\
\textit{correlazione discreta} \\
$R_a(f)=T\cdot\sum_{-\infty}^{+\infty}r_a(KT)e^{-j2\pi fKT}$ \\
\underline{\textit{Processi Aleatori noti}} \\
Rumore Termico $R_n(f)=2kTR$ \\
dove T=temperatura, R=resistenza, k=costante\\
$m_n=0$ $M_n=\infty$; \\
%Trasmissione numerica
\textbf{Trasmissione Numerica}; \\
$V_0$ valore più alto dell' impulso di decisione. 
$t=t_0 + KT$ istanti di decisione\\
$t_0$ tempo in cui abbiamo $V_0$ \\
SNR segnale-rumore; \\
$SNR=\frac{M_s}{M_n}$; \\ $M_s$ segnale, $M_n$ errore (spesso $\sigma^2$) \\ 
$SNR=\frac{E[{(a_k A)}^{2}]}{E[{n_k}^{2}]}$ \\
con $E[{(a_k A)}^{2}] = {A}^{2}\cdot \sum_{k=-\infty}^{+\infty}{P(a_k)\cdot {(a_k)}^{2}}$\\
\underline{\textit{PAM (equidistanti equiprobabili)}} \\
\textit{punti esterni}
$P_e = Q(\frac{V_o / 2}{\sigma}) $ \\
\textit{punti interni }
$P_e = 2Q(\frac{V_o / 2}{\sigma}) $ \\
\underline{\textit{PAM (formula generale)}} \\
\textit{Espressione del segnale PAM}
$s(t)=\sum_{k=-\infty}^{+\infty}a_k\cdot g(t-kT)$\\
$P_e=\sum{P[\widehat{a_k} \ne a_k | a_k=\alpha]\cdot P(a_k=\alpha)}$ \\
\underline{\textit{M-PAM (M numero di simboli) }} \\
$P_e = \frac{2(M-1)}{M}Q(\frac{V_0}{\sigma_n})$;\\
$SNR = \frac{V_0^2}{\sigma_n^2}\frac{M^2-1}{3}$; \\
$P_e = \frac{2(M-1)}{M}Q(\sqrt{\frac{SNR \cdot 3}{M^2-1}})$; \\
%Condizione di Intersybol Interference
\textbf{Condizione di no ISI}; \\
\textit{In tempo}
 $ q(t_{0}+kT)=
\begin{dcases}
    V_0\ne0 , k=0  \\
    V_0=0, k\ne0  \\
\end{dcases}
$ \\
\textit{In Frequenza}
$2B >= \frac{1}{T}$; \\
%Canali Trasmissivi
\textbf{Canali Trasmissivi}; \\
$T=T_b \log_{2}{M}$ ($T_b = \frac{1}{R_b}$ periodo di bit, T periodo di simbolo,
M = numero di simboli);\\
\underline{\textit{Radio}} \\
$P_R = P_T \cdot G_T \cdot G_R \cdot {(\frac{\lambda}{4\pi D})}^{2}$\\
dove $P_R$ potenza in recezione, $P_T$ potenza in trasmissione,
$G_T$ guadagno antenna in trasmissione,$G_R$ guadagno in antenna in recezione
$\lambda$ lunghezza d'onda, $D$ distanza\\
$f = \frac{c}{\lambda}$, $\lambda$ lunghezza d'onda, f frequenza portante
$c=3\cdot10^8$m/s  velocità della luce \\
$P_{out}=P_{in}\cdot K_0\cdot\frac{1}{D^2}$\\
$P_R >= P_{R_{\min}}$, $P_{R_{\min}}$ sensibilità ricevitore \\
$SNR=\frac{P_R}{P_N}$ \\
${(SNR)}_{dB}={(P_R)}_{dB}-{(P_N)}_{dB}$ \\
\underline{\textit{Cavo}} \\
$V_{out}=V_{IN}\cdot e^{-\beta L}$ con L=lunghezza cavo, \\
$\beta$ attenuazione specifica\\
$P_{out}=P_{IN}\cdot 10^{-\alpha D}$ \\
${(A_M)}_{dB}=\alpha\cdot L$ con $\alpha$ attenuazione\\
\underline{\textit{Fibra Ottica}} \\
$N_T = \frac{E_T}{\hbar\nu}$ numero fotoni in trasmissione \\
$\nu = \frac{c}{\lambda}$, $\lambda$ lunghezza d'onda \\
$c=3\cdot10^8$m/s  velocità della luce \\
$E_T = \sum_{x=0}^{n}{E_T(x)P(x)}$ x simboli \\
$E_T(x) = area[g_T(t)]\cdot x$ \\
$N_R = A_F\cdot N_T$ numero fotoni in recezione
$P_e=\frac{1}{2}e^{-2N_R}$ \\
\textit{con potenza massima/varianza }\\
$g_T(t)=\frac{A_T}{\sqrt{2\pi}\sigma_T}e^{\frac{-1}{2}{(\frac{t}{\sigma_T})}^{2}}$
\\ $P_T$ (o valore massimo)$=\frac{A_T}{\sqrt{2\pi}\sigma_T}$ \\
$area[g_T(t)] = A_T$ \\
\textit{filtro in trasmissione }\\
$h_F(t)=\frac{A_F}{\sqrt{2\pi}\sigma_F}e^{\frac{-1}{2}{(\frac{t}{\sigma_F})}^{2}}$ \\
$\sigma_F=\sigma_{F1} \cdot D$, $\sigma_{F1}$ dispersione km\\
\textit{segnale in recezione}\\ 
$g_R(t)=g_T(t)*h_F(t)$ dove \\
$g_R(t)=\frac{A_R}{\sqrt{2\pi}\sigma_R}e^{\frac{-1}{2}{(\frac{t}{\sigma_R})}^{2}}$
\\ con $A_R = A_T \cdot A_F$ e $\sigma_R = \sqrt{\sigma_F^2 + \sigma_T^2}$
$A_F = 10^{\frac{-\alpha D}{10}}$ $\alpha$ attenuazione km, D distanza fibra
oppure ${(A_F)}_{dB}=\alpha\cdot D$ \\
%Codifica
\textbf{Codifica}; \\
\underline{\textit{Codifica (3,1) (o codifica con fattore 3)}} \\
$1\cdot R_{b_{out}}=3\cdot R_{b_{in}}$\\
$R_{b_{out}}=\frac{R_{b_{in}}}{RateCodice}$\\
$RateCodice=\frac{k}{N}$
,k numero bit informazione, N totale bit del codice (nel caso 3,1) k=1,N=3 \\
\textit{Probabilità di errore}\\
$P_e=\sum_{i=\frac{N}{2}}^{N}\binom{N}{i}\cdot {P_e}^{i}\cdot {(1-P_e)}^{N-i}$\\
$i=\frac{N}{2}$ viene approssimato per eccesso.\\
\end{multicols*}
\end{document}
