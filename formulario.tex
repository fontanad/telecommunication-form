\documentclass[a4paper]{article}
\usepackage[T1]{fontenc}
\usepackage[utf8]{inputenc}
\usepackage[italian]{babel}
\usepackage{multicol}
\usepackage{color}
\usepackage{mathtools}
\usepackage{mathrsfs,amsmath}
\usepackage{geometry}
%margins
\geometry{a4paper,top=0cm,bottom=0.1cm,left=0cm,right=0cm,
    heightrounded,bindingoffset=0mm} 
%column lines
%ciao
\setlength{\columnseprule}{1pt}
\def\columnseprulecolor{\color{black}}
\begin{document}
\begin{multicols*}{3}
%Preliminari
\textbf{Preliminari} \\
Eulero; \\
$\cos(y)=\frac{e^{jy} + e^{-jy}}{2}$; \\
$\sin(y)=\frac{e^{jy} - e^{-jy}}{2}$; \\
\underline{\textit{Tempo Continuo}} \\
Area $\int_{-\infty}^{+\infty} x(t)$; \\
Energia $\int_{-\infty}^{+\infty} |x(t)|^2$; \\
Potenza 
$\lim_{T \to +\infty}\frac{1}{2T}\int_{-\infty}^{+\infty}|x(t)|^2$; \\
\underline{\textit{Tempo Discreto}} \\
Area $\sum_{n=-\infty}^{+\infty} x(nT) \cdot T$; \\
Energia $\sum_{n=-\infty}^{+\infty}|x(nT)|^2\cdot T$; \\
Potenza
$\lim_{N \to +\infty}{\frac{1}{2N + 1}\int_{n=-N}^{+N}|x(nT)|^2\cdot T}$; \\
%Segnali
\textbf{Segnali}; \\
Triangolo \\
$k\cdot triangle(t/\alpha) = rect(\alpha t) * rect(\alpha t)$; \\
Altezza Triangolo = k; \\
Area (triangle) = Area[rect] x Area[rect]; \\ 
$sinc^2(0) = 1$;\\
%Tasformate di fourier
\textbf{Fourier} \\
\underline{\textit{Tempo Continuo}} \\
$X(f) = \int_{-\infty}^{+\infty} x(t) e^{-j2\pi ft} dt$; \\
\underline{\textit{Proprietà}} \\
$ax(t) + by(t) \xrightarrow{\mathscr{F}} aX(f) + bY(f)$; \\
$x(-t) \xrightarrow{\mathscr{F}} X(-f)$; \\
$x^{*}(t) \xrightarrow{\mathscr{F}} X^{*}(f)$; \\
$x(t{\pm}t_0) \xrightarrow{\mathscr{F}} X(f)e^{{\pm}j2{\pi}f{t_0}}$; \\
$x(kt) \xrightarrow{\mathscr{F}} \frac{1}{|k|}X(\frac{f}{k})$; \\
$\frac{1}{|k|}X(\frac{t}{k}) \xrightarrow{\mathscr{F}} X(kf)$; \\
$x(t)e^{{\pm}j2{\pi}{f_0}t} \xrightarrow{\mathscr{F}} X(f{\mp}f_0)$; \\
$x(t)\cos(2{\pi}{f_0}t) \xrightarrow{\mathscr{F}} \frac{1}{2}[X(f - f_0) + X(f + f_0)]$; \\
$\int_{-\infty}^{+\infty}x(\tau)y(t - \tau) d\tau \xrightarrow{\mathscr{F}} X(f)Y(f)$; \\
$x(t)y(t) \xrightarrow{\mathscr{F}} \int_{-\infty}^{+\infty}X(a)Y(f - a) da$; \\
\underline{\textit{Trasformate Continue Note}} \\
$rect(t) \xrightarrow{\mathscr{F}} sinc(f)$; \\
$A \xrightarrow{\mathscr{F}} A\delta (f)$; \\
$e^{j2{\pi}f_{0}t} \xrightarrow{\mathscr{F}} \delta (f-f_0)$; \\
$A\cos(2{\pi}f_{0}t) \xrightarrow{\mathscr{F}} {\frac{A}{2}}[\delta(f-f_0) + \delta(f+f_0)]$; \\
$A\sin(2{\pi}f_{0}t) \xrightarrow{\mathscr{F}} {\frac{A}{2j}}[\delta(f-f_0) - \delta(f+f_0)]$; \\
$sinc(t) \xrightarrow{\mathscr{F}} rect(f)$; \\
$K\delta(t) \xrightarrow{\mathscr{F}} K$; \\
$e^{-\pi{t^2}} \xrightarrow{\mathscr{F}} e^{-\pi{f^2}}$; \\
\underline{\textit{Tempo Discreto}} \\
$X(f) = \sum_{k=-\infty}^{+\infty}{Tx(nT)e^{-j2\pi fnT}}$; \\
\underline{\textit{Trasformate Discrete Note}} \\
$\delta(nT) \xrightarrow{\mathscr{F}} 1$; \\
$\delta(nT - kT) \xrightarrow{\mathscr{F}} e^{-j2{\pi}{f_0}kT}$; \\
%Convoluzione
\textbf{Convoluzione} \\
\underline{\textit{Tempo Discreto}} \\
$\sum_{k=-\infty}^{+\infty}{Th(nT-Kt)x(nT)}$; \\
\underline{\textit{Tempo Continuo}} \\
TODO \\ 
\underline{\textit{Convoluzioni Note}} \\
$triangle(t) = rect(t) * rect(t)$; \\
%Modulazione 
\textbf{Modulazione} \\
$F_c>=2B$frequenza minima campionamento (B = 1/t)\\
%Variabili aleatorie
\textbf{Variabili Aleatorie}; \\
Varianza $\sigma^2=M_x-m_x^2$ \\
$M_x=\int_{-\infty}^{+\infty} a^2 f_x(a)$ (potenza statistica)\\
$m_x=\int_{-\infty}^{+\infty} a f_x(a)$; (media)\\
%conversione analogico digitake
\textbf{Conversione ADC} \\
$\Lambda_q$ segnale-errore di quantizzazione; \\
$L=2^b$ livelli quantizzati; \\
$\Lambda_q = L^2$ \\
${(\Lambda_q)}_{dB}$ SNR di quantizzazione; \\
${(\Lambda_q)}_{dB} = 6b$ b numero di bit; \\
${\Lambda}_q=\frac{M_s}{M_{e_q}}$ ($M_{e_q}$ errore quantizzazione 
,$M_s$ potenza statistica); \\
$P_{sat}=2Q(\frac{V_{\max}}{\sigma_x})$ (probabilità saturazione) \\
$R_b=F_c \cdot bit$ \\
$R_b$ bit rate; \\
$V_{\max}$ (dinamica) \\
\underline{\textit{Dinamica Normale}} \\
$V_{\max} = 4\sigma_a$ $\sigma_a$ deviazione standard;\\ 
%Processi aleatori
\textbf{Processi Aleatori}; \\
$r_y(t)$ correlazione; \\
$r_y(f) \xrightarrow{\mathscr{F}} R_y(f)$; \\ 
$R_y(f)$ densità spettrale di potenza; \\
$R_y(f) = R_x(f) |H(f)|^2$; \\
$M_y = r_y(0)$ oppure 
$M_y = \int_{-\infty}^{+\infty} R_y(f)$;\\
\underline{\textit{Processi Aleatori noti}} \\
Rumore Termico $R_n(f)=2kTR$ \\
$m_n=0$ $M_n=\infty$; \\
%Trasmissione numerica
\textbf{Trasmissione Numerica}; \\
SNR segnale-rumore; \\
$SNR=\frac{M_s}{M_n}$; \\ $M_s$ segnale, $M_n$ errore (spesso $\sigma^2$) \\ 
\underline{\textit{PAM}} \\
TODO (formula generica calcolo dell' errore) \\
\underline{\textit{M-PAM (M numero di coppie $\pm simbolo$) }} \\
$P_e = \frac{2(M-1)}{M}Q(\frac{V_0}{\sigma_n})$;\\
$SNR = \frac{V_0^2}{\sigma_n^2}\frac{M^2-1}{3}$; \\
$P_e = \frac{2(M-1)}{M}Q(\sqrt{\frac{SNR \cdot 3}{M^2-1}})$; \\
%Condizione di Intersybol Interference
\textbf{Condizione di no ISI}; \\
In tempo
 $ q(t_{0}+kT)=
\begin{dcases}
    V_0\ne0 , k=0  \\
    V_0=0, k\ne0  \\
\end{dcases}
$ \\
In Frequenza
$2B >= \frac{1}{T}$; \\
%Canali Trasmissivi
\textbf{Canali Trasmissivi}; \\
$T=T_b \log_{2}{M}$ ($T_b = \frac{1}{R_b}$, T periodo di simbolo,
M = numero di simboli);\\
\underline{\textit{Radio}} \\
TODO \\
\underline{\textit{Cavo}} \\
TODO \\
\underline{\textit{Fibra Ottica}} \\
$N_t=\frac{E_T}{\hbar\nu}$ $\nu =\frac{c}{\lambda}$,$\lambda$=lunghezza d'onda,
\\ $\hbar$ DA FINIRE \\
TODO \\
\end{multicols*}
\end{document}
